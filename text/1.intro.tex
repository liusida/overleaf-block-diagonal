
Recently, the number of papers related to Deep Learning is increasing rapidly.
arXiv.org acts as a central hub for important Deep Learning papers and opens to everyone.
Deep Learning papers are mainly distributed in two subcategories: cs.LG (Machine Learning) and cs.AI (Artificial Intelligence).
However, the number of papers in these two subcategories have reached over 90 thousands.
It is not possible for a human researcher to read all those papers.
Using Data Science approach to analyze them is inevitable.

We harvest the meta data from arXiv service and obtain the citation numbers from Semantic Scholar and Google Scholar.
We will see, although the total number of papers is big, the number of highly cited papers is relatively small.

We compare the authorships of the highly cited papers pairwisely to obtain a pairwise relationship matrix.
We reorder the relationship matrix by minimize the minimum linear arrangement (MinLA) cost function, so that the nodes that have relationship are arranged closer.
The optimization is done through a GPU augmented parallel hill climbing algorithm.
A reasonable good solution can be obtained in less than 10 minutes.

We visualize the relationship matrix and observe that clusters are formed, indicating there are some interesting structures of the research community.

Finally, we visualize the results as a sorted list of papers, with a corresponding background image to indicate the community structure.

