\subsection{Retrival}

arXiv is the largest hub for scientific papers.
It also supports the Open Archives Initiative (OAI), 
so a large volume of metadata can be downloaded through the OAI interface.
\footnote{More about arXiv OAI: https://arxiv.org/help/oa}

In April 2021, we download the metadata of 227,842 papers.
Our main interest field is Deep Learning.
Due to the computational resource constraints, we only processed two arXiv subcategories: cs.LG (Machine Learning) and cs.AI (Artificial Intelligence).
There are 93,911 papers in these two subcategories.

We continue to filter the papers by the citation numbers.
The citation number of a paper is an important indicator.
However, arXiv metadata does not include any information about citations,
and the accurate citation numbers are hard to obtain, 
so we get the citation numbers from two third-party websites.

The first data source is the Semantic Scholar (S2).
S2 provides additional metadata including the citation numbers of scientific papers.
\footnote{More about Semantic Scholar API: https://api.semanticscholar.org/}
We use the S2 citation numbers to filter out highly citated papers.
We keep 4,422 papers with S2 citation number large than or equal 100 as the highly cited papers.

The second data source is the Google Scholar.
Google Scholar also provides citation numbers on its website.
However, it does not support API retrival due to unspecified reason.
Getting citation number from the Goolge Scholar can only be done semi-automatically.

Two data sources both have its strength and shortage.
So we ensemble two parts together by taking the mean of the two.
Fig. \ref{fig:distribution} shows the distribution of the highly cited papers.
\footnote{Raw data of those 4422 papers can be downloaded at: http://}

\begin{figure}
    \centering
    \fbox{\rule[-.5cm]{4cm}{4cm} \rule[-.5cm]{4cm}{0cm}}
    \caption{Sample figure caption.}
    \label{fig:distribution}
\end{figure}

\subsection{Relationship Between Papers}

Now we have 4,422 highly cited papers.
We define a undirected weighted network $G(V,E)$, with its nodes $V$ to be the set of those papers, and its edges $E$ to be the set of the relationships between papers.

Our objective is to reveal the structure of the research community.
Based on this objective, we define the weights $w_{i,j}$ of the edge between papers based on its authors.
We divide the authors of a paper into two classes: $A$ the first author and the last author, $B$ other authors which is neither the first nor the last author.
\footnote{We manually correct several mistakes in metadata before dividing them into two classes.}
Class $A$ always at least contains one author, and class $B$ could be empty if there is no other authors.
We give a weight of 10 if class $A$ of two papers overlap.
Otherwise, we give a weight of 5 if class $A$ of one paper overlaps with $B$ of another paper.
Otherwise, we give a weight of 3 if class $B$ of two papers overlap.

Fig. \ref{fig:matrix_random_arrangement} shows the adjacent matrix of the constructed network $G$ with a random arrangement.

\begin{figure}
    \centering
    \fbox{\rule[-.5cm]{4cm}{4cm} \rule[-.5cm]{4cm}{0cm}}
    \caption{Sample figure caption.}
    \label{fig:matrix_random_arrangement}
\end{figure}
